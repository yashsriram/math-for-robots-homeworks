\documentclass[12pt]{article}
\usepackage[margin=1in]{geometry}
\usepackage{graphicx}
\usepackage{amsmath}
\usepackage{tikz}
\usepackage{hyperref}
\usepackage{enumitem}

\newcommand{\f}[1]{o_{#1}x_{#1}y_{#1}z_{#1}}
\newcommand{\fromlectures}{{\\ \color{blue} \hspace*{\fill}(from lecture slides)} \\}
\newcommand{\bydefn}{{\\ \color{blue} \hspace*{\fill}(by definition)} \\}
\newcommand{\given}{{\\ \color{blue} \hspace*{\fill}(given)} \\}
\newcommand{\rtp}{{\\ \color{blue} \hspace*{\fill}(required to prove)} \\}

\newcommand{\rx}[1]{\begin{bmatrix} 1 & 0 & 0 \\ 0 & cos(#1) & -sin(#1) \\ 0 & sin(#1) & cos(#1) \end{bmatrix}}
\newcommand{\ry}[1]{\begin{bmatrix} cos(#1) & 0 & sin(#1) \\ 0 & 1 & 0 \\ -sin(#1) & 0 & cos(#1) \end{bmatrix}}
\newcommand{\rz}[1]{\begin{bmatrix} cos(#1) & -sin(#1) & 0 \\ sin(#1) & cos(#1) & 0 \\ 0 & 0 & 1 \end{bmatrix}}

\title{CSci 5551 - HW2}
\author{Yashasvi Sriram Patkuri\\patku001@umn.edu}

\begin{document}
\maketitle
\pagebreak

\section{}
\subsection*{1.a}
$ k \equiv \begin{bmatrix} k_x \\ k_y \\ k_z \end{bmatrix} $ is a unit vector.
A rotation about k by an angle $\theta$ is considered.
\given

Rotation matrix corresponding to this rotation is
\[
  R \equiv
  \begin{bmatrix}
    k_x^2v\theta + c\theta & k_xk_yv\theta - k_zs\theta & k_xk_zv\theta + k_ys\theta\\
    k_xk_yv\theta + k_zs\theta & k_y^2v\theta + c\theta & k_yk_zv\theta - k_xs\theta \\
    k_xk_zv\theta - k_ys\theta & k_yk_zv\theta + k_xs\theta & k_z^2v\theta + c\theta
  \end{bmatrix}
\]
\rtp

The rotation matrix for a rotation about z-axis by angle $\theta$ is
\[
  R_{z,\theta} \equiv \rz{\theta}
\].
\fromlectures

\paragraph{Idea}
Consider a frame $ F_1 $ in which
\begin{enumerate}[nolistsep]
  \item The z-axis is aligned with what appears as k from global frame.
  \item The x and y axes are arbitrary perpendicular unit vectors in the plane perpendicular to its z-axis.
\end{enumerate}
By selecting an arbitrary y-axis perpendicular to z-axis, $ F_1 $ is completely known.
The rotation transformation matrix $ R_1^0 $ b/w global frame $ F_0 $ and $ F_1 $ is now completely known.

For a given vector $ v^0_a $ in $ F_0 $, we can get its representation $ v^1_a $ in $ F_1 $ using $ R_1^0 $.
We know the rotation matrix for rotation about z-axis by $\theta$.
We can use that to find rotated version of $ v^1_a $ say $ v^1_r $.
Finally we get the representation of $ v^1_r $ in $ F_0 $ viz. $ v^0_r $.

\subsubsection*{Finding $R_1^0$}
Let k be along z-axis of $ F_1 $. In general y-axis can be thought of as $ \begin{bmatrix} p_x \\ p_y \\ p_z \end{bmatrix} $.
As these are perpendicular unit vectors we have.
\[
  k_x^2 + k_y^2 + k_z^2 \equiv 1
\]
\[
  p_x^2 + p_y^2 + p_z^2 \equiv 1
\]
\[
  p_x * k_x + p_y * k_y + p_z * k_z \equiv 0
\]
As there is no restriction on the choice of y-axis we can choose the it to be in XY plane of $ F_0 $ i.e. it has the form $ \begin{bmatrix} p_x \\ p_y \\ 0 \end{bmatrix} $.
Therefore above equations become
\[
  k_x^2 + k_y^2 + k_z^2 \equiv 1
\]
\[
  p_x^2 + p_y^2 \equiv 1
\]
\[
  p_x * k_x + p_y * k_y \equiv 0
\]
Rewriting $ p_x $ in terms of $ p_y $ we have
\[
  p_x \equiv -p_y * \frac{k_y}{k_x}
\]
Substituting it back we get
\[
  (-p_y * \frac{k_y}{k_x})^2 + p_y^2 \equiv 1
\]
\[
  p_y^2 * \frac{k_y^2 + k_x^2}{k_x^2} \equiv 1
\]
We can choose the one of the square root. Therefore we get
\[
  p_y \equiv \frac{k_x}{\sqrt{k_x^2 + k_y^2}}
\]
\[
  p_x \equiv \frac{-k_y}{\sqrt{k_x^2 + k_y^2}}
\]
Therefore we have
\[
  p \equiv
  \begin{bmatrix}
  \frac{-k_y}{\sqrt{k_x^2 + k_y^2}} \\
  \frac{k_x}{\sqrt{k_x^2 + k_y^2}}  \\
  0
  \end{bmatrix}
\]
We can get the unit vector along x-axis $ t \equiv \begin{bmatrix} t_x \\ t_y \\ t_z \end{bmatrix} $ by taking a cross product
\[
  t \equiv p \times k
\]
\[
  t \equiv \begin{bmatrix} p_y k_z - p_z k_y \\ p_z k_x - p_x k_z \\ p_x k_y - p_y k_x \end{bmatrix}
\]
Substituting $ p_x, p_y, p_z $ in above equation we get
\[
  t \equiv
  \begin{bmatrix}
    \frac{k_1k_3}{\sqrt{k_1^2 + k_2^2}} \\
    \frac{k_2k_3}{\sqrt{k_1^2 + k_2^2}} \\
    -\sqrt{k_1^2 + k_2^2}
  \end{bmatrix}
\]
We have the rotation matrix
\[
  R_1^0 \equiv \begin{bmatrix} i_1i_0 & j_1i_0 & k_1i_0 \\ i_1j_0 & j_1j_0 & k_1j_0 \\ i_1k_0 & j_1k_0 & k_1k_0 \end{bmatrix}
\]
\fromlectures
Substituting columns we get
\[
  R_1^0 \equiv \begin{bmatrix} t & p & k \end{bmatrix}
\]
\[
  R_1^0 \equiv
  \begin{bmatrix}
  \frac{k_1k_3}{\sqrt{k_1^2 + k_2^2}} & \frac{-k_y}{\sqrt{k_x^2 + k_y^2}}  & k_x \\
  \frac{k_2k_3}{\sqrt{k_1^2 + k_2^2}} & \frac{k_x}{\sqrt{k_x^2 + k_y^2}}   & k_y \\
  -\sqrt{k_1^2 + k_2^2}               & 0 & k_z
  \end{bmatrix}
\]

\paragraph{Note} This formulation does not work iff $ k_x^2 + k_y^2 \equiv 0 $, i.e. $ k_x \equiv 0, k_y \equiv 0 $, which means $ k_z \equiv 1 $.
This case means that we are rotating about z-axis of $ F_0 $.
If we substitute values $ k_x \equiv 0, k_y \equiv 0, k_z \equiv 1 $ in given matrix in the question we get the rotation matrix about z-axis, thus proving the statement for this case.

We know that
\[
  R_1^0^{-1} \equiv R_1_0^T
\]
\fromlectures
Therefore we have
\[
  R_1^0^{-1} \equiv R_1_0^T \equiv
  \begin{bmatrix}
  \frac{k_1k_3}{\sqrt{k_1^2 + k_2^2}} & \frac{k_2k_3}{\sqrt{k_1^2 + k_2^2}} & -\sqrt{k_1^2 + k_2^2} \\
  \frac{-k_y}{\sqrt{k_x^2 + k_y^2}}  & \frac{k_x}{\sqrt{k_x^2 + k_y^2}}   & 0 \\
  k_x & k_y & k_z
  \end{bmatrix}
\]

\section{}
\section{}
\section{}
\section{}

\end{document}
