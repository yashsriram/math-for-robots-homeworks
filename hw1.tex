\documentclass[12pt]{article}
\usepackage[margin=1in]{geometry}
\usepackage{graphicx}
\usepackage{amsmath}
\usepackage{tikz}
\usepackage{hyperref}
\usepackage{enumitem}

\newcommand{\f}[1]{o_{#1}x_{#1}y_{#1}z_{#1}}
\newcommand{\fromslides}{{\\ \color{blue} \hspace*{\fill}(from lecture slides)} \\}
\newcommand{\bydefn}{{\\ \color{blue} \hspace*{\fill}(by definition)} \\}

\newcommand{\rx}[1]{\begin{bmatrix} 1 & 0 & 0 \\ 0 & cos(#1) & -sin(#1) \\ 0 & sin(#1) & cos(#1) \end{bmatrix}}
\newcommand{\ry}[1]{\begin{bmatrix} cos(#1) & 0 & sin(#1) \\ 0 & 1 & 0 \\ -sin(#1) & 0 & cos(#1) \end{bmatrix}}
\newcommand{\rz}[1]{\begin{bmatrix} cos(#1) & -sin(#1) & 0 \\ sin(#1) & cos(#1) & 0 \\ 0 & 0 & 1 \end{bmatrix}}

\title{CSci 5551 - HW1}
\author{Yashasvi Sriram Patkuri\\patku001@umn.edu}

\begin{document}
\maketitle
\pagebreak

\section{}
Let $ \f{\frac{1}{2}} $ represent the coordinate system formed by rotating $ \f{0} $ by $ \frac{\pi}{2} $ radians about the x-axis.
Let this rotation be denoted by $ R_1 $. Using basic matrices
\begin{equation}
  R_1 \equiv \rx{\frac{\pi}{2}} \equiv \begin{bmatrix} 1 & 0 & 0 \\ 0 & 0 & -1 \\ 0 & 1 & 0 \end{bmatrix}
\end{equation}
\fromslides

$ \f{1} $ represents the coordinate system formed by rotating $ \f{\frac{1}{2}} $ by$ \frac{\pi}{2} $ radians about the fixed frame Y-axis.
Let this rotation be denoted by $ R_2 $. Using basic matrices
\begin{equation}
  R_2 \equiv \ry{\frac{\pi}{2}} \equiv \begin{bmatrix} 0 & 0 & 1 \\ 0 & 1 & 0 \\ -1 & 0 & 0 \end{bmatrix}
\end{equation}
\fromslides

For two consecutive fixed frame rotations denoted by $ R_1, R_2 $ rotation matrices, the composite rotation matrix, say R $ \equiv $ $ R_2 * R_1 $

\begin{equation}
  R
  \equiv R_2 * R_1
  \equiv \begin{bmatrix} 0 & 0 & 1 \\ 0 & 1 & 0 \\ -1 & 0 & 0 \end{bmatrix} * \begin{bmatrix} 1 & 0 & 0 \\ 0 & 0 & -1 \\ 0 & 1 & 0 \end{bmatrix}
  \equiv \begin{bmatrix} 0 & 1 & 0 \\ 0 & 0 & -1 \\ -1 & 0 & 0 \end{bmatrix}
\end{equation}
\fromslides

The sketches of frames $ \f{1} $, $ \f{\frac{1}{2}} $ and $ \f{1} $ are drawn in figure \ref{fig:q1}.

\begin{figure}[htpb]
  \centering
  \begin{tikzpicture}[x=1cm, y=1cm, z=-0.6cm]
      \draw [->] (0,0,0) -- (4,0,0) node [right] {$x_0$};
      \draw [->] (0,0,0) -- (0,4,0) node [left] {$y_0$};
      \draw [->] (0,0,0) -- (0,0,4) node [left] {$z_0$};
  \end{tikzpicture}

  \begin{tikzpicture}[x=1cm, y=1cm, z=-0.6cm]
      \draw [->] (0,0,0) -- (4,0,0) node [right] {$x_{\frac{1}{2}}$};
      \draw [->] (0,0,0) -- (0,0,4) node [left] {$y_{\frac{1}{2}}$};
      \draw [->] (0,0,0) -- (0,-4,0) node [left] {$z_{\frac{1}{2}}$};
  \end{tikzpicture}

  \begin{tikzpicture}[x=1cm, y=1cm, z=-0.6cm]
      \draw [->] (0,0,0) -- (0,0,-4) node [left] {$x_1$};
      \draw [->] (0,0,0) -- (4,0,0) node [right] {$y_1$};
      \draw [->] (0,0,0) -- (0,-4,0) node [left] {$z_1$};
  \end{tikzpicture}
\caption{Rotations by $ \frac{\pi}{2} $ radians about fixed X and Y axes consecutively}
  \label{fig:q1}
\end{figure}

\pagebreak
\section{}

Given
\begin{equation}
  R_{12} \equiv \begin{bmatrix} 1 & 0 & 0 \\ 0 & \frac{1}{2} & -\frac{\sqrt{3}}{2} \\ 0 & \frac{\sqrt{3}}{2} & \frac{1}{2} \end{bmatrix}
\end{equation}
\begin{equation}
  R_{13} \equiv \begin{bmatrix} 0 & 0 & -1 \\ 0 & 1 & 0 \\ 1 & 0 & 0 \end{bmatrix}
\end{equation}

For any physical vector p, let $ p^{i} $ be its representation in ith coordinate system
\begin{equation}
  \label{eq:21}
  p^{2} \equiv R_{21} * p^{1}
\end{equation}
\begin{equation}
  \label{eq:22}
  p^{1} \equiv R_{13} * p^{3}
\end{equation}
\bydefn

Identities \ref{eq:21} and \ref{eq:22} imply
\begin{equation}
  \label{eq:23}
  p^{2} \equiv R_{21} * R_{13} * p^{3}
\end{equation}

But
\begin{equation}
  \label{eq:24}
  p^{2} \equiv R_{23} * p^{3}
\end{equation}
\bydefn

From identities \ref{eq:23} and \ref{eq:24}
\begin{equation}
  \label{eq:25}
  R_{23} \equiv R_{21} * R_{13}
\end{equation}

But
\begin{equation}
  \label{eq:26}
  R_{21} \equiv R_{12}^{T}
\end{equation}
\fromslides

Identities \ref{eq:25} and \ref{eq:26} imply
\begin{equation}
  \label{eq:27}
  R_{23} \equiv R_{12}^T * R_{13}
  \equiv \begin{bmatrix} 1 & 0 & 0 \\ 0 & \frac{1}{2} & \frac{\sqrt{3}}{2} \\ 0 & -\frac{\sqrt{3}}{2} & \frac{1}{2} \end{bmatrix} * \begin{bmatrix} 0 & 0 & -1 \\ 0 & 1 & 0 \\ 1 & 0 & 0 \end{bmatrix}
  \equiv \begin{bmatrix} 0 & 0 & -1 \\ \frac{\sqrt{3}}{2} & \frac{1}{2} & 0 \\ \frac{1}{2} & -\frac{\sqrt{3}}{2} & 0 \end{bmatrix}
\end{equation}

\pagebreak
\section{}

Let us represent the Z-X-Z angles by $ \theta_1, \theta_2, \theta_3 $ instead of $ \phi, \theta, \psi $ respectively for simplicity.

First we build rotation matrix corresponding to each Eular angle.
Using basic rotation matrices we get
\begin{equation}
  \label{eq:31}
  R_{\theta_1,z} \equiv \rz{\theta_1}
\end{equation}
\begin{equation}
  \label{eq:32}
  R_{\theta_2,x} \equiv \rx{\theta_2}
\end{equation}
\begin{equation}
  \label{eq:33}
  R_{\theta_3,z} \equiv \rz{\theta_3}
\end{equation}

Eular angles correspond to consecutive current frame rotations.
For consecutive current frame rotations $ R_1, R_2, R_3 $, the composite rotation matrix is say R is given by
\begin{equation}
  \label{eq:34}
  R \equiv R_1 * R_2 * R_3
\end{equation}
\fromslides

Substituting equations \ref{eq:31}, \ref{eq:32}, \ref{eq:33} in \ref{eq:34} we get,
\begin{equation}
  \label{eq:35}
  R_{ZXZ} \equiv R_{\theta_1,z} * R_{\theta_2,x} * R_{\theta_3,z}
\end{equation}
where $ R_{ZXZ} $ represents rotation matrix corresponding to the set of Z-X-Z eular angles.

\[
  R_{ZXZ} \equiv \rz{\theta_1} * \rx{\theta_2} * \rz{\theta_3}
\]
\[
  R_{ZXZ} \equiv \begin{bmatrix} c1 & -s1.c2 & s1.s2 \\ s1 & c1.c2 & -c1.s2 \\ 0 & s2 & c2 \end{bmatrix} * \rz{\theta_3}
\]
Where ci represents $ cos(\theta_i) $ and si represents $ sin(\theta_i) $
\begin{equation}
  \label{eq:36}
  R_{ZXZ} \equiv \begin{bmatrix} c1.c3 - s1.c2.s3 & -c1.s3 - s1.c2.c3 & s1.s2 \\ s1.c3 + c1.c2.s3 & -s1.s3 + c1.c2.c3 & -c1.s2 \\ s2.s3 & s2.c3 & c2 \end{bmatrix}
\end{equation}

Equation \ref{eq:36} is the rotation matrix corresponding to Z-X-Z Eular angles.

\subsection{$sin(\theta) = 0$}
For us $\theta_2$ represents $\theta$.
$sin(\theta_2) = 0$ implies $cos(\theta_2) = 1$. Substituting these in equation \ref{eq:36} we get
\begin{equation}
  \label{eq:37}
  R_{ZXZ} \equiv \begin{bmatrix} c1.c3 - s1.s3 & -c1.s3 - s1.c3 & 0 \\ s1.c3 + c1.s3 & -s1.s3 + c1.c3 & 0 \\ 0 & 0 & 1 \end{bmatrix}
\end{equation}
But
\begin{equation}
  \label{eq:38}
  sin(a + b) = sin(a)cos(b) + cos(a)sin(b)
\end{equation}
\begin{equation}
  \label{eq:39}
  cos(a + b) = cos(a)cos(b) - sin(a)sin(b)
\end{equation}

From equations \ref{eq:37}, \ref{eq:38} and \ref{eq:39} we can write
\begin{equation}
  \label{eq:310}
  R_{ZXZ} \equiv \rz{\theta_1 + \theta_3}
\end{equation}
Therefore, when $sin(\theta) = 0$ the composite rotation matrix is identical to rotation about Z-axis by $ \theta_1 + \theta_3 $.
{\color{red} Does it lose a degree of freedom? Gimbal lock?}.

\end{document}
